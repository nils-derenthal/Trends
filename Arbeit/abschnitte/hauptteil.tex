% ca. 50%

\section{Hauptteil}

In diesem Abschnitt wird der zentrale Inhalt der Arbeit behandelt.
Ausgehend von grundlegenden Konzepten der Typensysteme wird schrittweise erläutert, 
wie diese Konzepte auf die Struktur und Definition moderner Web-APIs übertragen werden können.
Ziel ist es, ein Verständnis dafür zu schaffen, wie Typisierung zur Fehlervermeidung und zur 
strukturierten Entwicklung von Schnittstellen beiträgt.

\subsection{Begriffserläuterungen}

\subsubsection{Typen und Typsystem}

Die Grundsätzliche Idee eines Typsystems ist die Vermeidung von Fehlern zur
Laufzeit eines Programmes. Typischerweise wird hierfür ein Algorithmus eingesetzt welcher überprüft,
ob alle Typen, die innerhalb eines Programmtextes verwendet werden, konsistent sind.

Typen können als eine Menge von möglichen Werten gesehen werden. So ist beispielsweise die Menge
für den Typen von Ganzzahlen (oft \texttt{integer} oder \texttt{int}) im Fall von $32$ bit
$[-2^{31},2^{31}-1]$.
Einer der simpelsten Typen sind Wahrheitswerte (oft \texttt{boolean} oder \texttt{bool}) welche aus 
der Menge $\{\text{true}, \text{false}\}$ bestehen.

In vielen Sprachen gibt es primitive Typen dessen Wertemengen genau ein Element umfassen. In Java ist dies Beispielsweise
\texttt{null} und in TypeScript existieren \texttt{null} und \texttt{undefined}. \footnote{Auch wenn \texttt{void} einen zusätzlichen unären Typen darstellt wird dieser hier ignoriert}. Diese Typen können in kombination mit Summen (siehe \ref{sec:algebraic-types}) die mögliche Abwesenheit eines Wertes darstellen.
In Java Beispielsweise kann jeder nicht primitive Typ implizit den Wert \texttt{null} besitzen.

Oft stellen Sprachen Möglichkeiten bereit aus diesen, wie auch weiteren,
als \textit{primitiv} bezeichneten Typen, komplexere Strukturen zu bilden um viele Anwendungsfälle
abzudecken.

Ein Typsystem beschreibt die Regeln einer Programmiersprache zur Bildung, Kombination und Überprüfung von diesen Typen.

\subsubsection{Algebräische Typen} \label{sec:algebraic-types}

In der Typentheorie ist oft von algebräischen Typen die Rede. Damit sind sogenannte Summen- und Produkttypen gemeint.
Auch wenn diese in der Funktionalen Programmierung oft Rekursiv sind, werden im folgenden ausschließlich nicht-rekursive
Strukturen beachtet. \cite{https://doi.org/10.1002/spe.1058}

Summen (auch \texttt{Vereinigung} oder \texttt{Union}) stellen eine Auswahl zwischen mehreren Typen da.\cite{288374} Beispielsweise kann eine Antwort einer API das tatsächliche
Ergebnis einer Anfrage \textit{oder} eine Fehlermeldung zurückgeben. In der Typentheorie wird für Summentypen das $+$ Zeichen oder $|$ verwendet.
\lstinline{Antwort = Fehler | TatsaechlicheAntwort} wäre eine Möglichkeit das Oben genannte Beispiel darzustellen.
Mögliche Werte für \lstinline{Antwort} kommen nun aus der \textit{Vereinigung} der Mengen von den Typen 
\lstinline{Fehler} und \lstinline{TatsaechlicheAntwort}.

Produkte (auch \texttt{Kartesisches Produkt}) sind ein Zusammenschluss von mehreren Typen.
Als Notation für Produkte wird entweder $\times$ oder $\&$ benutzt.
Wenn eine Anfrage Beispielsweise Informationen zu einem Nutzeraccount liefert könnten
diese eine ID, einen Namen sowie eine E-mail Adresse Enthalten.
Dies kann folgendermaßen dargestellt werden: \lstinline{NutzerInfos = ID & Name & Email}
In der Praxis, beispielsweise in Sprachen wie TypeScript, sind Produktypen oft benannt, sie sind also Schlüssel-Wert Paare.\cite{https://doi.org/10.1002/spe.1058}

Mit Hilfe dieser beiden Operationen lassen sich bis auf wenige Ausnahmen, auf die nicht weiter eingegangen wird, alle Typen darstellen.\cite{10.1145/2633628.2633634}

\subsection{Schnittstellendefinition von REST APIs}

Schnittstellen werden oft durch sogenannte DTOs (Data Transfer Objects) definiert. Diese nehmen oft die Form von einfachen Produkttypen an.
Um den Transfer von Daten über das Netzwerk möglich zu machen wird oft von Serialisierung gebrauch gemacht um die Daten in einem genormten 
Format zu senden.\cite{Monday2003} Eins der häufigsten Formate hierbei ist JSON (JavaScript Object Notation).

Im folgenden wird ein Beispiel aus der Gitlab REST API Betrachtet. Der Endpunkt \texttt{/projects/:id/repository/files/:filepath} erlaubt es
Benutzer*innen eine Datei aus einem Repository abzufragen. Die Erhaltene Antwort von GitLab folgt immer dem selben Schema:
\begin{lstlisting}[language=json]
{
  "file_name": "file.hs",
  "size": 1476,
  ...
  "last_commit_id": "570e7b2abdd848b...",
  "execute_filemode": false
}
\end{lstlisting}

Dieses Schema lässt sich als ein Produkt der einzelnen Schlüssel-Wert Paare Darstellen: \\
\lstinline[language=json]|file_name=string & size=number & ...|\\

Es ist möglich das eine Schnittstelle optionale Felder besitzt, welche mit einer Summe 
des unären Typen (bespielsweise \texttt{undefined}) dargestellt wird.
\footnote{Auf den Unterschied zwischen optionalen Feldern und erforderlichen optionalen Feldern wird hier nicht weiter eingegangen.}.

Beispielsweise könnte eine Referenz auf eine assoziierte Merge Request existieren, dies muss aber nicht der Fall sein:

Eine Vereinfachte Darstellung, des Oben genannten Beispiels mit Optionalem Feld 
in einem JSON-ähnlichen Format könnte wiefolgt aussehen:

\begin{lstlisting}[language=json]
{
  "file_name": string,
  "size": number,
  "related_merge_request": number | undefined
  ...
}
\end{lstlisting}

\subsubsection{API Schemas \& OpenAPI}

Zur formalen Definition und Dokumentation\cite{10.1145/2577080.2577098} von REST APIs haben sich sogenannte API-Schemas etabliert.
Diese beschreiben die möglichen Anfragen und Antworten, die eine API erwartet bzw. zurückgibt.
Ein weit verbreiteter Standard zur Beschreibung solcher Schnittstellen ist OpenAPI \footnote{ehemals Swagger}.\cite{Ed-douibiHamza2018OATt}\cite{BognerJustus2023DRAd}

Ein OpenAPI-Dokument beschreibt typischerweise unteranderem:
\begin{itemize}
    \item Pfade (Endpoints) der API
    \item HTTP-Methoden (GET, POST, etc.)
    \item Anfrageparameter und deren Typen
    \item Erwartete Antworttypen und HTTP-Statuscodes
\end{itemize}

im folgenden soll hierbei primär die Beschreibung der Afragen und Antworten betrachtet werden.
Ein zentrales Konzept ist das sogenannte Schema, das die Struktur dieser Daten beschreibt. 
Die Schemata basieren in OpenAPI auf JSON.

Ein einfaches Beispiel für ein OpenAPI Schema, das dem bereits zuvor betrachteten GitLab Beispiel ähnelt, könnte wie folgt aussehen:

\begin{lstlisting}[language=json]
{
  "type": "object",
  "properties": {
    "file_name": { "type": "string" },
    "size": { "type": "integer" },
    "last_commit_id": { "type": "string" },
    "execute_filemode": { "type": "boolean" }
  },
  "required": ["file_name", "size"]
}
\end{lstlisting}

Hierbei ist ersichtlich, dass es sich um einen Produkttyp handelt,
dessen Felder unterschiedliche primitive Typen besitzen.
Felder, die nicht im \texttt{required}-Array gelistet sind, sind optional, was, wie zuvor beschrieben,
typentheoretisch einer Summe mit einem unären Typen wie \texttt{undefined} entspricht.

Neben der Möglichkeit eine API so zu dokumentieren können Schemata wie diese auch dazu verwendet werden
um diese Maschinell zu analysieren und zu transformieren.

\subsubsection{Typgenerierung}

Aus OpenAPI-Schemas lassen sich automatisch Typdefinitionen für Programmiersprachen generieren, 
was die Konsistenz zwischen API-Dokumentation und Implementierung sicherstellt.
Dies ist möglich bei Projekten die eine eigene Backend- und Frontend-Struktur besitzen, sowie als auch
bei Projekten die ausschließlich eine externe API benutzen welche ein OpenAPI Schema bereitstellt.

Für TypeScript ist das Tool \texttt{openapi-typescript} verfügbar.
Aus dem zuvor gezeigten Schema wird folgenden Typ erzeugt:
\begin{lstlisting}[language=json]
interface FileInfo {
  file_name: string;
  size: number;
  last_commit_id: string | undefined;
  execute_filemode: boolean | undefined;
}
\end{lstlisting}
Optionale Felder, also die, die nicht in dem \texttt{required} array existieren, werden als optional deklariert.

Neben der Generierung einer Typdefinition aus einem Schema ermöglichen OpenAPI-Tools auch die Erstellung von
Schemata aus bestehenden APIs.
Zum Beispiel kann im Spring Framework für Java das Tool springdoc-openapi ein Schema aus einem 
RestController generieren.
Wenn das generierte Schema vom Frontend zur Erstellung seiner Typdefinition verwendet wird ergibt sich auch
hier eine klare Definition, ohne das das Schema händisch erstellt werden muss.

\subsubsection{Limitationen in Bezug auf Typen}

Obwohl OpenAPI, wie aufgezeigt die Möglichkeit gibt Typen aus definitierten Schnittstellen zu bilden, 
gibt es auch einige wichtige Limitationen bei der Übersetzung.

Ein zentrales Problem ist, dass OpenAPI über das Attribut \texttt{format} zusätzliche semantische Einschränkungen
auf primitive Typen spezifizieren kann, die über die reine Typinformation hinausgehen.
Beispielsweise kann ein Feld vom Typ \texttt{string} das Format \texttt{date-time} tragen, 
was bedeutet, dass nur Zeichenketten im ISO 8601 Zeitformat gültig sind. 
Auch für \texttt{integer} oder \texttt{number} existieren Formate wie 
\texttt{int32}, \texttt{int64} oder \texttt{float}. 
Diese Formate definieren eingeschränkte Wertebereiche oder bestimmte Darstellungsformen.

Das Problem hierbei ist, dass viele Programmiersprachen, insbesondere solche mit struktureller Typisierung wie TypeScript,
diese Formatinformationen nicht direkt in ihre Typensysteme übersetzen können. 
Während das Tooling den Typ korrekt als \texttt{string} oder \texttt{number} übernimmt, 
geht die mit dem Format verbundene Einschränkung dabei verloren. 
Der erzeugte Typ vermittelt also nur einen Teil der in OpenAPI spezifizierten Bedeutung, 
was zu einem Verlust an Präzision führen kann.

Ein weiteres Beispiel betrifft Wertebereiche, 
die über die \texttt{minimum}- und \texttt{maximum}-Eigenschaften eines Schemas definiert werden.
Auch hier kann OpenAPI präzise definieren, dass zum Beispiel ein Zahlenwert zwischen 0 und 100 liegen muss. 
In vielen Zielsprachen lässt sich dies jedoch nicht direkt typisieren, 
da es keine eingebauten Mechanismen für bereichsbegrenzte Typen gibt.
Entsprechend müssen solche Einschränkungen entweder in Validierungslogik separat abgebildet oder manuell dokumentiert werden,
was wiederum potenzielle Inkonsistenzen schafft.

\subsubsection{Herausforderungen automatischer Typgenerierung}

Trotz der offensichtlichen Vorteile automatisierter Typgenerierung, wie der Reduktion manueller Fehler
und der Sicherstellung von Konsistenz zwischen Dokumentation und Implementierung und automatischen Tests\cite{9159071}
bringt dieser Ansatz auch einige Herausforderungen mit sich. 
Ein zentrales Problem stellt der sogenannte Schema-Drift dar:
Wenn sich das OpenAPI-Schema ändert, aber die generierten Typen im Frontend nicht aktualisiert werden, 
kann dies zu schwer auffindbaren Laufzeitfehlern führen. 
Zudem besteht bei generierten Typen die Gefahr einer Fehlinferenz, 
etwa wenn das Schema nicht präzise genug formuliert ist oder vereinfachte Annahmen über optionale Felder getroffen werden.
In größeren Projekten kann es außerdem zu Konflikten zwischen manuell gepflegten und automatisch erzeugten Typdefinitionen kommen,
insbesondere wenn Entwickler*innen lokal Anpassungen vornehmen, 
die beim nächsten Generierungslauf überschrieben werden. 
In solchen Fällen ist eine klare Tooling-Strategie sowie ein bewusstes Versions- und Änderungsmanagement unerlässlich, 
um Konsistenz und Wartbarkeit langfristig sicherzustellen.

\subsection{Persönliche Reflexion}

In meiner bisherigen Arbeit als Softwareentwickler bin ich immer wieder auf die Herausforderungen gestoßen,
die inkonsistente Schnittstellen mit sich bringen. 
Diese Probleme reichen von Missverständnissen in der Kommunikation zwischen verschiedenen Systemen, 
bis hin zu erhöhtem Wartungsaufwand und einer schlechteren Benutzererfahrung. 
Der Umgang mit solchen Schnittstellen kann durchaus mühsam sein, 
da kleine Änderungen oder Fehler schwerwiegende Auswirkungen auf die gesamte Systemarchitektur haben können.

In diesem Kontext habe ich OpenAPI als eine interessante Möglichkeit kennengelernt, 
um Schnittstellen klarer und standardisierter zu gestalten. 
OpenAPI ermöglicht es, Schnittstellen strukturiert und maschinenlesbar zu definieren, was viele Vorteile mit sich bringt.
Durch die Dokumentation von APIs in einer klaren, standardisierten Form können potenzielle Inkonsistenzen frühzeitig erkannt 
und behoben werden, was den Entwicklungsprozess insgesamt effizienter und weniger fehleranfällig macht.

Trotz der vielen Vorteile, die OpenAPI mit sich bringt, bleibt ein gewisser kritischer Blick bestehen. 
Die Definition von APIs erfordert eine sorgfältige Planung und kann bei komplexen Systemen schnell unübersichtlich werden.
Zudem besteht die Gefahr, dass bei einer rein automatisierten Generierung der Schnittstellen die Qualität und Flexibilität leidet,
wenn Entwickler nicht ausreichend in den Prozess eingebunden sind. 
Es bleibt also eine Herausforderung, das Potenzial von OpenAPI voll auszuschöpfen, 
ohne die Kontrolle über die Feinheiten der Schnittstellen und deren Anpassungsfähigkeit zu verlieren.

Insgesamt sehe ich OpenAPI als eine vielversprechende Methode zur Verbesserung der Konsistenz und Wartbarkeit von Schnittstellen,
auch wenn ich die Notwendigkeit für eine kritische und bewusste Implementierung nicht aus den Augen verlieren möchte. 
Die Balance zwischen Automatisierung und manuellem Feingefühl bleibt dabei entscheidend,
um langfristig stabile und leistungsfähige Systeme zu entwickeln.
