\section{Hauptteil}
% Eigentlicher Inhalt der Arbeit mit geeigneter Einteilung in Unterkapitel und ggf. Querbezügen zu anderen Themen des Seminars und der persönlichen Reflexion des Trends (siehe Folien). \\ \\
% Der Hauptteil sollte ca. 50\% vom Umfang der Arbeit ausmachen

\subsection{Begriffserläuterungen}

\subsubsection{Typen und Typsystem}

Die Grundsätzliche Idee eines Typsystems ist die verhinderung von verhinderung von Fehlern zur
Laufzeit eines Programmes. Typischerweise wird hierfür ein Algorithmus eingesetzt welcher überprüft,
dass alle Typen die innerhalb eines Programmtextes verwendet werden untereinander stimmig sind.

Typen können als eine Menge von möglichen Werten gesehen werden. So ist beispielsweise die Menge
für den Typen von Ganzzahlen (oft \texttt{integer} oder \texttt{int}) im Fall von $32$ bit 
$[-2^{31},2^{31}-1]$.
Einer der simpelsten Typen sind Wahrheitswerte (oft \texttt{boolean} oder \texttt{bool}) welche aus 
der Menge $\{\text{true}, \text{false}\}$ bestehen.

Oft stellen Sprachen Möglichkeiten bereit aus diesen, wie auch weiteren, 
als \textit{primitiv} bezeichneten Typen, komplexere Strukturen zu bilden um kompliziertere Anwendungsfälle
abzudecken. Das einfachste Konzept stellt hierbei ein \textit{struct} da, welches einen Zusammengesetzen
Datentypen aus verschiedenen anderen Zusammenfast. Dies wird in \ref{sec:structural_types} genauer dargestellt.


Das Typesystem einer Sprache ist der Oberbegriff dafür wie Typen gebildet werden können und wie die richtigkeit dieser
überprüft wird, falls das der Fall ist.

\subsubsection{Statische und dynamische Typisierung}
\subsubsection{Starke und schwache Typisierung}

\subsection{Unterschiede der Typsysteme diverser Sprachen}

Im folgenden werden die Typsysteme von C, Python und Haskell betrachtet und ihre Unterschiede dargestellt.

\subsubsection{Strukturelle Typen (C)}\label{sec:structural_types}
\subsubsection{Dynamische Typen (Python)} % sicher?
\subsubsection{Abhängige Typen (Haskell)}

Algebraische Datentypen?

\subsection{Relevanz für die Webentwicklung}

\subsubsection{Schnittstellendefinition}

- Rest API?

- JSON?

\subsubsection{Vor- und Nachteile}

\subsection{Persönliche Reflexion}
