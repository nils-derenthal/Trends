% ca. 10%

\section{Zusammenfassung und Ausblick}

\subsection{Zusammenfassung}

In der Arbeit wurden einige theoretische Grundlagen von Typsystemen in Bezug auf Web APIs, 
sowie Beispielmodellierungen dargestellt.
Darauf basierend wurde das Tool OpenAPI vorgestellt, welches zur Definition von Schnittstellen genutzt werden kann.
Darüber hinaus wurden weitere Möglichkeiten aufgezeigt; Primär die automatisierte generierung von den
zuvor beschriebenen Typen.
Abschließend wurden Risiken und Herrausforderungen bei der Nutzung von OpenAPI beschrieben.

\subsection{Kritische Reflektion} 

Im Nachhinein lässt sich einiges an Verbesserungspotential sehen. Einer der größten Faktoren ist die Zeitaufteilung,
welche nicht Ideal abgelaufen ist. Einerseits habe ich zu spät angefangen und somit einen Großteil der Arbeit gegen
Ende der geplanten Phase geschrieben, andererseits hätte ich mir konkrete Zwischenziele setzen, und so strukturierter
arbeiten können.

Bei der Recherche hätte ich mir mehr Zeit lassen können und erst Stichpunkte herrausarbeiten können anstatt, dass ich
einfach drauf los schreibe. Dies ist auch darauf Zurückzuführen, dass ich ein persönliches Interesse an dem
bearbeiteten Thema hatte, und so eventuell voreingenommen in die Arbeit gegangen bin.
Hier könnte es Hilfreich sein die Recherche von einer "unwissenderen" Perspektive durchzuführen um neutralere
Ergebnisse zu erhalten.

Außerdem wäre eine bessere Planung der Struktur der Arbeit hilfreich gewesen, da ich so das tatsächliche Thema der
Arbeit während dem Schreiben noch verändert habe. Dies ist nicht Ideal und ist auch in Bezug auf die Fragen,
die mit der Arbeit beantwortet werden sollen nicht perfekt. Es wäre vermutlich besser gewesen, 
hätte ich mehr Zeit darauf angewendet, die Zielsetzung der Arbeit strukturiert durchzuführen und erst dann angefangen
zu schreiben, wie das auch mit der Literaturrecherche der Fall gewesen ist.

Insgesamt bin ich mit meiner Ausarbeitung zufrieden, auch wenn es einzelne Punkte gibt, die ich verbessern könnte.
Dies werde ich mir als Ziel für zukünftige Arbeiten setzen.

\subsection{Ausblick}

Basierend auf der Arbeit wäre es möglich verschiedene weitergehende Themen zu behandeln. 
Einerseits kann untersucht werden wie effektiv Typsysteme für den Anwendungsfall der APIs sind, 
anderseits können aber auch Thematisch tiefere Themen beleuchtet werden. 
Beispielsweise könnten weitere Generierungsmöglichkeiten aus OpenAPI Schemas beleuchtet werden 
(Automatisierte API Zugriffe durch KI Agenten, Maschinelles Testen, Dokumentation, etc.).

Auch in Bezug auf Typsysteme und die Typentheorie lassen sich weitere Themenkomplexe finden, wie
beispielsweise Summentypen für Systemnahe Programmierung verwendet werden können oder die Einsatzmöglichkeiten
von Typen als Richtlinie für Generative KI Systeme in der Programmierung.

Auch wenn dies in der Arbeit bereits angeschnitten wurde, wäre es des Weiteren möglich die Frage zu beantworten,
wie sinnvoll der Einsatz von diesen Tools tatsächlich ist, und welche empirischen Beweise oder Studien hierfür sprechen.
Man könnte sich kritisch mit Typensystemen allgemein beschäftigen und schauen, ob der Einsatz nicht vielleicht
sogar hinderlich ist.
