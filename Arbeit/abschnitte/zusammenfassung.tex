% ca. 10%

\section{Zusammenfassung und Ausblick}

\subsection{Zusammenfassung}
%Was sind die zentralen Ergebnisse/Erkenntnisse eurer Arbeit?

\subsection{Kritische Reflektion} 

Im Nachhinein lässt sich einiges an Verbesserungspotential sehen. Einer der größten Faktoren ist die Zeitaufteilung,
welche nicht Ideal abgelaufen ist. Einerseits habe ich zu spät angefangen und somit einen Großteil der Arbeit gegen
Ende der geplanten Phase geschrieben, andererseits hätte ich mir konkrete Zwischenziele setzen, und so strukturierter
arbeiten können.

Bei der Recherche hätte ich mir mehr Zeit lassen können und erst Stichpunkte herrausarbeiten können anstatt, dass ich
einfach drauf los schreibe. Dies ist auch darauf Zurückzuführen, dass ich ein persönliches Interesse an dem
bearbeiteten Thema hatte, und so eventuell voreingenommen in die Arbeit gegangen bin.
Hier könnte es Hilfreich sein die Recherche von einer "unwissenderen" Perspektive durchzuführen um neutralere
Ergebnisse zu erhalten.

\subsection{Ausblick}

Basierend auf der Arbeit wäre es möglich verschiedene weitergehende Themen zu behandeln. 
Einerseits kann untersucht werden wie effektiv Typsysteme für den Anwendungsfall der APIs sind, 
anderseits können aber auch Thematisch tiefere Themen beleuchtet werden. 
Beispielsweise könnten weitere Generierungsmöglichkeiten aus OpenAPI Schemas beleuchtet werden 
(Automatisierte API Zugriffe durch KI Agenten, Maschinelles Testen, Dokumentation, etc.).

Auch in Bezug auf Typsysteme und die Typentheorie lassen sich weitere Themenkomplexe finden, wie
beispielsweise Summentypen für Systemnahe Programmierung verwendet werden können oder die Einsatzmöglichkeiten
von Typen als Richtlinie für Generative KI Systeme in der Programmierung.
