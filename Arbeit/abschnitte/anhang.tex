\section{Anhang}
% Der Anhang dient als ergänzender Teil der Arbeit und  zählt nicht zur Zeichenbeschränkung.

% optional, falls ihr viele Tabellen nutzt
% \listoftables

% optional, falls ihr viele Abbildungen nutzt
% \listoffigures

\newpage

% Die Liste der verwendeten Literatur, außer Literatur die mit \mybibexclude{qullen-id} markiert wurde
\printbibliography[notcategory=fullcited]

\newpage
\newgeometry{
    right=0.7in,
    left=0.7in            
}

% Literaturliste zum Rechercheprotokoll
% \fullcite{qullen-id} gibt die entsprechende Quelle komplett aus. 
% \mybibexclude{qullen-id} verhindert, dass die entsprechende Quelle im Literaturverzeichnis auftaucht

\def\uncite#1{\fullcite{#1}\mybibexclude{#1}}

\begin{tabularx}{\textwidth}{|X|X|}
    \hline
    \textbf{Suchergebnisse} & \textbf{Notizen} \\ \hline \hline
    \fullcite{10.1145/2577080.2577098} & Effektivität von Typen(namen) als Dokumentation \\ \hline
    \fullcite{9159071} & Automatisierte Tests mit hilfe von OpenAPI \\ \hline
    \uncite{10.1145/3643991.3644932} & Datensammlung von Quelloffenen OpenAPI Schemata (Ungenutzt) \\ \hline
    \uncite{TZAVARAS2023100675} & OpenAPI als Tool für WoT APIs (Ungenutzt) \\ \hline
    \fullcite{Ed-douibiHamza2018OATt} & OpenAPI als Möglichkeit Dokumentation in Form von UML Diagrammen zu generieren \\ \hline
    \fullcite{10.1145/2633628.2633634} & Beweis für die Darstellbarkeit von Algebräischen Typen als Summen von Produkten \\ \hline
    \fullcite{288374} & Definition von Summentypen \\ \hline
    \fullcite{https://doi.org/10.1002/spe.1058} & Tool zum definieren von Algebräischen Datentypen in C \\ \hline
    \fullcite{BognerJustus2023DRAd} & Verständlichkeit von REST APIs \\ \hline
    \fullcite{Monday2003} & Data Transfer Objects als Brücke zwischen Systemen \\ \hline
    \caption{Literaturliste zum Rechercheprotokoll}
\end{tabularx}
