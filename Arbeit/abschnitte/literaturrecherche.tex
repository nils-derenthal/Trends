% ca. 20%

\section{Systematische Literaturrecherche}
%Dieses Kapitel beschreibt in knapper Form das Vorgehen zur Literaturrecherche. Genaueres ist in \cite{keele2007guidelines} zu finden. Dieses Kapitel sollte ca. 20\% vom Umfang der Arbeit ausmachen. Nachfolgend die Leitfragen für die Recherche aus der Einführungsveranstaltung: \\ \\


% Definition eines Rechercheziels - Was möchte nach der Recherche gelernt haben? Schützen Atemschutzmasken vor einer Corona-Infektion? \\ \\
% Definition relevanter Orte/Quellen für Literatur \\
% Bild-Zeitung, Science, The Lancet, … \\ \\
% Definition relevanter Suchbegriffe \\
% Corona, COVID-19, Maske, Mask, FFP-2 \\ \\
% Durchführung der Recherche mit Protokoll \\
% Datum, Quelle, Suchbegriffe, Ergebnisliste  \\ \\
% Bewertung der Ergebnisse \\
% Aktualität, Glaubwürdigkeit, Beitrag zum Rechercheziel

\subsection{Rechercheziel}

Die Zielsetzung der Literaturrecherche besteht darin, den aktuellen Stand der Forschung und Praxis
im Bereich der Typisierung von Web-APIs systematisch zu erfassen und zu analysieren.
Dabei sollen insbesondere folgende Aspekte untersucht werden:

\begin{itemize}
\item Die Rolle und Bedeutung von Typensystemen bei der Modellierung und Beschreibung von API-Schnittstellen, insbesondere im Kontext moderner Web-APIs.
\item Die Anwendung und Umsetzung typentheoretischer Konzepte wie Produkt- und Summentypen in der API-Definition und deren formale Spezifikation durch Standards wie OpenAPI.
\item Die Möglichkeiten und Grenzen automatischer Typgenerierung aus API-Schemata sowie die damit verbundenen Herausforderungen in der Praxis.
\item Bestehende Ansätze und Werkzeuge zur Sicherstellung der Konsistenz zwischen API-Dokumentation und Implementierung.
\end{itemize}

Die Literaturrecherche soll somit eine Grundlage schaffen, 
um die Relevanz und den Nutzen von Typensystemen bei der API-Modellierung zu bewerten,
und die in der Arbeit dargestellten Lösungsansätze einzuordnen.

\subsection{Relevante Quellen für die Literatur}

Zur Quellensuche wurde vor allem Google Scholar herrangezogen, welche die Suche nach wissenschaftlichen Quellen
vereinfacht.
Auf Basis der Suchen wurden die folgenden Quellen / Veröffentlichungsorte von wissenschaftlichen Papieren aufgrund ihrer Relevanz
auf Bezug zu dem Rechercheziel identifiziert.

\subsubsection{ACM Digital Library}

Die ACM Digital Library (ACMDL) ist eine geeignete Quelle für wissenschaftliche Arbeiten 
zu Themen aus der Informatik, da sie peer-reviewte Publikationen aus diesem Bereich bereitstellt.
Sie bietet Zugang zu aktueller Forschung und anerkannten Konferenzen,
die regelmäßig Beiträge zu relevanten Aspekten des Themas veröffentlichen.
Die Inhalte sind wissenschaftlich fundiert und werden von Fachleuten der Community erstellt. \cite{ACM}

\subsubsection{IEEE Xplore Digital Library}

Die IEEE Xplore Digital Library bietet eine verlässliche Grundlage für wissenschaftliche Arbeiten
in diesem Bereich.
Sie umfasst eine große Auswahl an Fachartikeln, Konferenzbeiträgen und technischen Standards 
mit Schwerpunkt auf Informatik und verwandten Disziplinen.
Besonders relevant ist sie aufgrund ihres Fokus auf praxisnahe und anwendungsorientierte Forschung,
die aktuelle Entwicklungen im Themenfeld umfassend abbildet.
Die Inhalte stammen aus etablierten Fachkreisen und unterliegen strengen Qualitätsstandards. \cite{IEEE}

\subsubsection{Blogartikel}

Blogposts bieten eine wichtige Ressource zur schnellen Informationsgewinnung und praktischen Anwendung. 
Entwicklerblogs und technische Beiträge von Unternehmen oder Fachleuten liefern häufig praxisnahe Anleitungen,
Beispielcodes und Erfahrungsberichte, die in der wissenschaftlichen Literatur oft fehlen oder nur stark abstrahiert vorhanden sind. 
Dadurch ermöglichen Blogposts einen unmittelbaren Zugang zu neuen Technologien, Best Practices und aktuellen Entwicklungen.
Gleichzeitig sind jedoch Risiken für die wissenschaftliche Arbeit erkenntlich:
Die Inhalte unterliegen meist keiner formalen Qualitätskontrolle, sind nicht immer langfristig verfügbar und 
reflektieren oft subjektive Sichtweisen einzelner Entwickler*innen.
Für eine fundierte wissenschaftliche Nutzung müssen solche Quellen daher kritisch bewertet und 
durch etablierte Literatur oder offizielle Dokumentationen ergänzt werden.

\subsubsection{Offizielle Dokumentation}

Die offizielle Dokumentation stellt in der Informatik eine der zuverlässigsten Quellen für die Nutzung und Implementierung 
von Technologien dar.
Sie wird in der Regel von den Entwickler*innen der jeweiligen Software oder Plattformen erstellt und bietet detaillierte, 
präzise Informationen zu Funktionen, Parametern und Anwendungsmöglichkeiten. 
Da offizielle Dokumentationen oft direkt aus dem Quellcode und den Designentscheidungen der Entwickler*innen abgeleitet werden, 
gelten sie als die autoritativste und genaueste Informationsquelle.
Zudem bieten sie eine stabile und langfristig verfügbare Referenz,
die regelmäßig aktualisiert wird, um mit neuen Versionen und Änderungen Schritt zu halten.
Ein Risiko lässt sich allerdings darin sehen, dass eine voreingenommene Haltung gegenüber der präsentierten Software existiert.
Deshalb muss auch hier mit Hilfe von unabhängigen Quellen belegt werden wie Beispielsweise Effektiv eine Software ist.

\subsection{Relevante Suchbegriffe}

Auch wenn die Arbeit im Deutschen Formuliert ist, wird ein Großteil der Recherche im Englischen
durchgeführt, da in der Informatik die meiste Literatur Englischsprachig ist, und es so warscheinlicher ist,
dass Quellen gefunden werden.

\subsubsection{Typisierung}

In Bezug auf die Typisierung der Schnittstellen wurden die folgenden
Recherchebegriffe herrausgearbeitet:

\begin{itemize}
\item type systems web APIs
\item static typing API interfaces
\item product and sum types
\item type safety web APIs
\item typing REST bodys
\end{itemize}

Die Suchergebnisse zu diesen Begriffen waren oft Zielführend, allerdings gab es auch viele
Paper, die nicht die gesuchten Informationen geliefert haben.
Einige von den gefundenen Ergebnissen sind zwar älter, können aber trotzdem als
relevant eingestuft werden, da sie sich mit Typensystemen befassen, welche ein
sehr theoretisches und sich nicht änderndes Thema darstellen.
Bei den Quellen zur Webentwicklung wurde darauf geachtet, dass diese aktuelle Entwicklungen darstellen
und nicht veraltet sind.

\subsubsection{OpenAPI}

Auf Basis der Ergebnisse über die Typisierung wurde die Suche nach Informationen zu OpenAPI
mit den folgenden Suchbegriffen durchgeführt:

\begin{itemize}
\item OpenAPI
\item OpenAPI type systems
\item OpenAPI type safety
\item OpenAPI generation
\item OpenAPI endpoints
\end{itemize}

Auch hier hat die Suche sich auf neuere Paper fokussiert, was kein Problem dargestellt hat, da
das spezielle Tooling noch nicht sehr alt ist.

\subsubsection{Herrausforderungen und Risiken von OpenAPI}

Weiterführend wurden für die Suche nach Risiken und Herrausforderungen bei der Nutzung 
von OpenAPI Suchbegriffe definiert:

\begin{itemize}
\item OpenAPI issues
\item OpenAPI schema drift
\item OpenAPI risks
\end{itemize}

Hier wurde besonders darauf geachtet, dass die Paper von einer neutralen Quelle kommen, da
nur so die Integrität sichergestellt werden kann.

\subsection{Bewertung der Quellen}

Die gefundenen Quellen zeigen eine hohe Vertraulichkeit auf, da die meisten von bekannten Universiäten oder
Instituten, sowie in renomierten Magazinen oder Konferenzen veröffentlicht wurden.
Auch auf den Inhalt bezogen wurde eine hohe Qualität der Quellen festgestellt, da
viele Verweise auf vorherige Arbeiten existieren und der Inhalt neutral und Sachgerecht geschrieben ist.

Insgesamt werden die gefundenen Quellen als qualitativ hochwertig befunden und können somit in der Arbeit verwendet werden.

