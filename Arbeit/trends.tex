\documentclass{article}
\usepackage{glossaries}
\usepackage{hyperref}
\usepackage{tabularx}
\usepackage{ltablex}
\usepackage[pass]{geometry}

% https://www.overleaf.com/learn/latex/German
\usepackage[utf8]{inputenc}
\usepackage[T1]{fontenc}
\usepackage{csquotes}

\usepackage[backend=biber, style=alphabetic]{biblatex}

\addbibresource{bibfile.bib}
% https://tex.stackexchange.com/questions/111363/exclude-fullcite-citation-from-bibliography#111375
% erstellt einen Befehl, der ermöglicht, dass Quellen mit \fullcite{...} komplett zitiert werden können ohne im Literaturverzeichnis aufgeführt zu werden
\DeclareBibliographyCategory{fullcited}
\newcommand{\mybibexclude}[1]{\addtocategory{fullcited}{#1}}

\usepackage[ngerman]{babel}
% LaTeX trennt bekanntlich Wörter automatisch. Unbekannte Wörter können wie folgt beigebracht werden.
% \usepackage{hyphenat}
% \hyphenation{Mathe-matik wieder-gewinnen}

\usepackage{subcaption}
\usepackage{graphicx}
\graphicspath{{./assets}}

\usepackage{float}
\usepackage{listings}
\usepackage{xcolor}

\definecolor{amber}{rgb}{1.0, 0.49, 0.0}
\definecolor{green}{rgb}{0.4, 1.0, 0.7}

\lstdefinelanguage{Types}{
    keywords={export, interface}
}

\lstdefinelanguage{json}{
    basicstyle=\normalfont\ttfamily,
    numberstyle=\scriptsize,
    breaklines=true,
    showstringspaces=false,
    string=[db]{"},
    stringstyle=\color{green!50!black},
    morestring=[s][\color{black}]{\ \ "}{":},
    keywordstyle=\color{blue},
    keywords={true,false,null,export,interface},
    literate=
     *{0}{{{\color{red}0}}}{1}
      {1}{{{\color{red}1}}}{1}
      {2}{{{\color{red}2}}}{1}
      {3}{{{\color{red}3}}}{1}
      {4}{{{\color{red}4}}}{1}
      {5}{{{\color{red}5}}}{1}
      {6}{{{\color{red}6}}}{1}
      {7}{{{\color{red}7}}}{1}
      {8}{{{\color{red}8}}}{1}
      {9}{{{\color{red}9}}}{1}
      {.}{{{\color{red}.}}}{1}
      {:}{{{\color{gray}{:}}}}{1}
      {,}{{{\color{gray}{,}}}}{1}
      {\{}{{{\color{gray}{\{}}}}{1}
      {\}}{{{\color{gray}{\}}}}}{1}
      {[}{{{\color{gray}{[}}}}{1}
      {]}{{{\color{gray}{]}}}}{1},
}

\lstset{frame=tb,
  language=Types,
  aboveskip=3mm,
  belowskip=3mm,
  showstringspaces=false,
  columns=flexible,
  basicstyle={\small\ttfamily},
  numbers=left,
  numberstyle=\tiny\color{blue},
  keywordstyle=\color{blue},
  commentstyle=\color{dkgreen},
  stringstyle=\color{green},
  identifierstyle=\color{amber},
  breaklines=true,
  breakatwhitespace=true,
  tabsize=4
}

\begin{document}

\begin{titlepage}
    \begin{center}
        \vspace*{1cm}

        \LARGE Seminararbeit im Seminar Trends der Softwaretechnik \\
        \large Sommersemester 2025

        \vspace{1.5cm}

        \huge
        \textbf{Typsichere Schnittstellendefinition von API-Endpunkten}

        \vspace{3cm}

        \large
        \textbf{Nils Derenthal} (7217566) \\
        \def\email{nils.derenthal002@stud.fh-dortmund.de}\tt\href{mailto:\email}{\email} \\
        Informatik Dual

        \vfill
    \end{center}
\end{titlepage}

\tableofcontents

\newpage


\section{Einleitung} 

Typesysteme, amirite?


\subsection{Grundlagen} 
Was muss jemand mit eurem Wissensstand zusätzlich wissen, um euer Thema zu verstehen.

\subsection{Problemstellung} 
Welches Problem (innerhalb der Softwaretechnik) adressiert der Trend? Warum ist das Problem ein Problem? Wie ist der Lösungsansatz für das Problem? 

\subsection{Ziel der Arbeit} 
Was hat der/die Leser*in von dieser Arbeit? Was soll er/sie nach dem Lesen gelernt haben?

% ca. 20%

\section{Systematische Literaturrecherche}
%Dieses Kapitel beschreibt in knapper Form das Vorgehen zur Literaturrecherche. Genaueres ist in \cite{keele2007guidelines} zu finden. Dieses Kapitel sollte ca. 20\% vom Umfang der Arbeit ausmachen. Nachfolgend die Leitfragen für die Recherche aus der Einführungsveranstaltung: \\ \\


% Definition eines Rechercheziels - Was möchte nach der Recherche gelernt haben? Schützen Atemschutzmasken vor einer Corona-Infektion? \\ \\
% Definition relevanter Orte/Quellen für Literatur \\
% Bild-Zeitung, Science, The Lancet, … \\ \\
% Definition relevanter Suchbegriffe \\
% Corona, COVID-19, Maske, Mask, FFP-2 \\ \\
% Durchführung der Recherche mit Protokoll \\
% Datum, Quelle, Suchbegriffe, Ergebnisliste  \\ \\
% Bewertung der Ergebnisse \\
% Aktualität, Glaubwürdigkeit, Beitrag zum Rechercheziel

\subsection{Rechercheziel}

Die Zielsetzung der Literaturrecherche besteht darin, den aktuellen Stand der Forschung und Praxis
im Bereich der Typisierung von Web-APIs systematisch zu erfassen und zu analysieren.
Dabei sollen insbesondere folgende Aspekte untersucht werden:

\begin{itemize}
\item Die Rolle und Bedeutung von Typensystemen bei der Modellierung und Beschreibung von API-Schnittstellen, insbesondere im Kontext moderner Web-APIs.
\item Die Anwendung und Umsetzung typentheoretischer Konzepte wie Produkt- und Summentypen in der API-Definition und deren formale Spezifikation durch Standards wie OpenAPI.
\item Die Möglichkeiten und Grenzen automatischer Typgenerierung aus API-Schemata sowie die damit verbundenen Herausforderungen in der Praxis.
\item Bestehende Ansätze und Werkzeuge zur Sicherstellung der Konsistenz zwischen API-Dokumentation und Implementierung.
\end{itemize}

Die Literaturrecherche soll somit eine Grundlage schaffen, 
um die Relevanz und den Nutzen von Typensystemen bei der API-Modellierung zu bewerten,
und die in der Arbeit dargestellten Lösungsansätze einzuordnen.

\subsection{Relevante Quellen für die Literatur}

Zur Quellensuche wurde vor allem Google Scholar herrangezogen, welche die Suche nach wissenschaftlichen Quellen
vereinfacht.
Auf Basis der Suchen wurden die folgenden Quellen / Veröffentlichungsorte von wissenschaftlichen Papieren aufgrund ihrer Relevanz
auf Bezug zu dem Rechercheziel identifiziert.

\subsubsection{ACM Digital Library}

Die ACM Digital Library (ACMDL) ist eine geeignete Quelle für wissenschaftliche Arbeiten 
zu Themen aus der Informatik, da sie peer-reviewte Publikationen aus diesem Bereich bereitstellt.
Sie bietet Zugang zu aktueller Forschung und anerkannten Konferenzen,
die regelmäßig Beiträge zu relevanten Aspekten des Themas veröffentlichen.
Die Inhalte sind wissenschaftlich fundiert und werden von Fachleuten der Community erstellt. \cite{ACM}

\subsubsection{IEEE Xplore Digital Library}

Die IEEE Xplore Digital Library bietet eine verlässliche Grundlage für wissenschaftliche Arbeiten
in diesem Bereich.
Sie umfasst eine große Auswahl an Fachartikeln, Konferenzbeiträgen und technischen Standards 
mit Schwerpunkt auf Informatik und verwandten Disziplinen.
Besonders relevant ist sie aufgrund ihres Fokus auf praxisnahe und anwendungsorientierte Forschung,
die aktuelle Entwicklungen im Themenfeld umfassend abbildet.
Die Inhalte stammen aus etablierten Fachkreisen und unterliegen strengen Qualitätsstandards. \cite{IEEE}

\subsubsection{Blogartikel}

Blogposts bieten eine wichtige Ressource zur schnellen Informationsgewinnung und praktischen Anwendung. 
Entwicklerblogs und technische Beiträge von Unternehmen oder Fachleuten liefern häufig praxisnahe Anleitungen,
Beispielcodes und Erfahrungsberichte, die in der wissenschaftlichen Literatur oft fehlen oder nur stark abstrahiert vorhanden sind. 
Dadurch ermöglichen Blogposts einen unmittelbaren Zugang zu neuen Technologien, Best Practices und aktuellen Entwicklungen.
Gleichzeitig sind jedoch Risiken für die wissenschaftliche Arbeit erkenntlich:
Die Inhalte unterliegen meist keiner formalen Qualitätskontrolle, sind nicht immer langfristig verfügbar und 
reflektieren oft subjektive Sichtweisen einzelner Entwickler*innen.
Für eine fundierte wissenschaftliche Nutzung müssen solche Quellen daher kritisch bewertet und 
durch etablierte Literatur oder offizielle Dokumentationen ergänzt werden.

\subsubsection{Offizielle Dokumentation}

Die offizielle Dokumentation stellt in der Informatik eine der zuverlässigsten Quellen für die Nutzung und Implementierung 
von Technologien dar.
Sie wird in der Regel von den Entwickler*innen der jeweiligen Software oder Plattformen erstellt und bietet detaillierte, 
präzise Informationen zu Funktionen, Parametern und Anwendungsmöglichkeiten. 
Da offizielle Dokumentationen oft direkt aus dem Quellcode und den Designentscheidungen der Entwickler*innen abgeleitet werden, 
gelten sie als die autoritativste und genaueste Informationsquelle.
Zudem bieten sie eine stabile und langfristig verfügbare Referenz,
die regelmäßig aktualisiert wird, um mit neuen Versionen und Änderungen Schritt zu halten.
Ein Risiko lässt sich allerdings darin sehen, dass eine voreingenommene Haltung gegenüber der präsentierten Software existiert.
Deshalb muss auch hier mit Hilfe von unabhängigen Quellen belegt werden wie Beispielsweise Effektiv eine Software ist.

\subsection{Relevante Suchbegriffe}

Auch wenn die Arbeit im Deutschen Formuliert ist, wird ein Großteil der Recherche im Englischen
durchgeführt, da in der Informatik die meiste Literatur Englischsprachig ist, und es so warscheinlicher ist,
dass Quellen gefunden werden.

\subsubsection{Typisierung}

In Bezug auf die Typisierung der Schnittstellen wurden die folgenden
Recherchebegriffe herrausgearbeitet:

\begin{itemize}
\item type systems web APIs
\item static typing API interfaces
\item product and sum types API
\item type safety web APIs
\end{itemize}



% ca. 50%

\section{Hauptteil}

In diesem Abschnitt wird der zentrale Inhalt der Arbeit behandelt.
Ausgehend von grundlegenden Konzepten der Typensysteme wird schrittweise erläutert, 
wie diese Konzepte auf die Struktur und Definition moderner Web-APIs übertragen werden können.
Ziel ist es, ein Verständnis dafür zu schaffen, wie Typisierung zur Fehlervermeidung und zur 
strukturierten Entwicklung von Schnittstellen beiträgt.

\subsection{Begriffserläuterungen}

\subsubsection{Typen und Typsystem}

Die Grundsätzliche Idee eines Typsystems ist die Vermeidung von Fehlern zur
Laufzeit eines Programmes. Typischerweise wird hierfür ein Algorithmus eingesetzt welcher überprüft,
ob alle Typen, die innerhalb eines Programmtextes verwendet werden, konsistent sind.

Typen können als eine Menge von möglichen Werten gesehen werden. So ist beispielsweise die Menge
für den Typen von Ganzzahlen (oft \texttt{integer} oder \texttt{int}) im Fall von $32$ bit
$[-2^{31},2^{31}-1]$.
Einer der simpelsten Typen sind Wahrheitswerte (oft \texttt{boolean} oder \texttt{bool}) welche aus 
der Menge $\{\text{true}, \text{false}\}$ bestehen.

In vielen Sprachen gibt es primitive Typen dessen Wertemengen genau ein Element umfassen. In Java ist dies Beispielsweise
\texttt{null} und in TypeScript existieren \texttt{null} und \texttt{undefined}. \footnote{Auch wenn \texttt{void} einen zusätzlichen unären Typen darstellt wird dieser hier ignoriert}. Diese Typen können in kombination mit Summen (siehe \ref{sec:algebraic-types}) die mögliche Abwesenheit eines Wertes darstellen.
In Java Beispielsweise kann jeder nicht primitive Typ implizit den Wert \texttt{null} besitzen.

Oft stellen Sprachen Möglichkeiten bereit aus diesen, wie auch weiteren,
als \textit{primitiv} bezeichneten Typen, komplexere Strukturen zu bilden um viele Anwendungsfälle
abzudecken.

Ein Typsystem beschreibt die Regeln einer Programmiersprache zur Bildung, Kombination und Überprüfung von diesen Typen.

\subsubsection{Algebräische Typen} \label{sec:algebraic-types}

In der Typentheorie ist oft von algebräischen Typen die Rede. Damit sind sogenannte Summen- und Produkttypen gemeint.
Auch wenn diese in der Funktionalen Programmierung oft Rekursiv sind, werden im folgenden ausschließlich nicht-rekursive
Strukturen beachtet. \cite{https://doi.org/10.1002/spe.1058}

Summen (auch \texttt{Vereinigung} oder \texttt{Union}) stellen eine Auswahl zwischen mehreren Typen da.\cite{288374} Beispielsweise kann eine Antwort einer API das tatsächliche
Ergebnis einer Anfrage \textit{oder} eine Fehlermeldung zurückgeben. In der Typentheorie wird für Summentypen das $+$ Zeichen oder $|$ verwendet.
\lstinline{Antwort = Fehler | TatsaechlicheAntwort} wäre eine Möglichkeit das Oben genannte Beispiel darzustellen.
Mögliche Werte für \lstinline{Antwort} kommen nun aus der \textit{Vereinigung} der Mengen von den Typen 
\lstinline{Fehler} und \lstinline{TatsaechlicheAntwort}.

Produkte (auch \texttt{Kartesisches Produkt}) sind ein Zusammenschluss von mehreren Typen.
Als Notation für Produkte wird entweder $\times$ oder $\&$ benutzt.
Wenn eine Anfrage Beispielsweise Informationen zu einem Nutzeraccount liefert könnten
diese eine ID, einen Namen sowie eine E-mail Adresse Enthalten.
Dies kann folgendermaßen dargestellt werden: \lstinline{NutzerInfos = ID & Name & Email}
In der Praxis, beispielsweise in Sprachen wie TypeScript, sind Produktypen oft benannt, sie sind also Schlüssel-Wert Paare.\cite{https://doi.org/10.1002/spe.1058}

Mit Hilfe dieser beiden Operationen lassen sich bis auf wenige Ausnahmen, auf die nicht weiter eingegangen wird, alle Typen darstellen.\cite{10.1145/2633628.2633634}

\subsection{Schnittstellendefinition von REST APIs}

Schnittstellen werden oft durch sogenannte DTOs (Data Transfer Objects) definiert. Diese nehmen oft die Form von einfachen Produkttypen an.
Um den Transfer von Daten über das Netzwerk möglich zu machen wird oft von Serialisierung gebrauch gemacht um die Daten in einem genormten 
Format zu senden.\cite{Monday2003} Eins der häufigsten Formate hierbei ist JSON (JavaScript Object Notation).

Im folgenden wird ein Beispiel aus der Gitlab REST API Betrachtet. Der Endpunkt \texttt{/projects/:id/repository/files/:filepath} erlaubt es
Benutzer*innen eine Datei aus einem Repository abzufragen. Die Erhaltene Antwort von GitLab folgt immer dem selben Schema:
\begin{lstlisting}[language=json]
{
  "file_name": "file.hs",
  "size": 1476,
  ...
  "last_commit_id": "570e7b2abdd848b...",
  "execute_filemode": false
}
\end{lstlisting}

Dieses Schema lässt sich als ein Produkt der einzelnen Schlüssel-Wert Paare Darstellen: \\
\lstinline[language=json]|file_name=string & size=number & ...|\\

Es ist möglich das eine Schnittstelle optionale Felder besitzt, welche mit einer Summe 
des unären Typen (bespielsweise \texttt{undefined}) dargestellt wird.
\footnote{Auf den Unterschied zwischen optionalen Feldern und erforderlichen optionalen Feldern wird hier nicht weiter eingegangen.}.

Beispielsweise könnte eine Referenz auf eine assoziierte Merge Request existieren, dies muss aber nicht der Fall sein:

Eine Vereinfachte Darstellung, des Oben genannten Beispiels mit Optionalem Feld 
in einem JSON-ähnlichen Format könnte wiefolgt aussehen:

\begin{lstlisting}[language=json]
{
  "file_name": string,
  "size": number,
  "related_merge_request": number | undefined
  ...
}
\end{lstlisting}

\subsubsection{API Schemas \& OpenAPI}

Zur formalen Definition und Dokumentation\cite{10.1145/2577080.2577098} von REST APIs haben sich sogenannte API-Schemas etabliert.
Diese beschreiben die möglichen Anfragen und Antworten, die eine API erwartet bzw. zurückgibt.
Ein weit verbreiteter Standard zur Beschreibung solcher Schnittstellen ist OpenAPI \footnote{ehemals Swagger}.\cite{Ed-douibiHamza2018OATt}\cite{BognerJustus2023DRAd}

Ein OpenAPI-Dokument beschreibt typischerweise unteranderem:
\begin{itemize}
    \item Pfade (Endpoints) der API
    \item HTTP-Methoden (GET, POST, etc.)
    \item Anfrageparameter und deren Typen
    \item Erwartete Antworttypen und HTTP-Statuscodes
\end{itemize}

im folgenden soll hierbei primär die Beschreibung der Afragen und Antworten betrachtet werden.
Ein zentrales Konzept ist das sogenannte Schema, das die Struktur dieser Daten beschreibt. 
Die Schemata basieren in OpenAPI auf JSON.

Ein einfaches Beispiel für ein OpenAPI Schema, das dem bereits zuvor betrachteten GitLab Beispiel ähnelt, könnte wie folgt aussehen:

\begin{lstlisting}[language=json]
{
  "type": "object",
  "properties": {
    "file_name": { "type": "string" },
    "size": { "type": "integer" },
    "last_commit_id": { "type": "string" },
    "execute_filemode": { "type": "boolean" }
  },
  "required": ["file_name", "size"]
}
\end{lstlisting}

Hierbei ist ersichtlich, dass es sich um einen Produkttyp handelt,
dessen Felder unterschiedliche primitive Typen besitzen.
Felder, die nicht im \texttt{required}-Array gelistet sind, sind optional, was, wie zuvor beschrieben,
typentheoretisch einer Summe mit einem unären Typen wie \texttt{undefined} entspricht.

Neben der Möglichkeit eine API so zu dokumentieren können Schemata wie diese auch dazu verwendet werden
um diese Maschinell zu analysieren und zu transformieren.

\subsubsection{Typgenerierung}

Aus OpenAPI-Schemas lassen sich automatisch Typdefinitionen für Programmiersprachen generieren, 
was die Konsistenz zwischen API-Dokumentation und Implementierung sicherstellt.
Dies ist möglich bei Projekten die eine eigene Backend- und Frontend-Struktur besitzen, sowie als auch
bei Projekten die ausschließlich eine externe API benutzen welche ein OpenAPI Schema bereitstellt.

Für TypeScript ist das Tool \texttt{openapi-typescript} verfügbar.
Aus dem zuvor gezeigten Schema wird folgenden Typ erzeugt:
\begin{lstlisting}[language=json]
interface FileInfo {
  file_name: string;
  size: number;
  last_commit_id: string | undefined;
  execute_filemode: boolean | undefined;
}
\end{lstlisting}
Optionale Felder, also die, die nicht in dem \texttt{required} array existieren, werden als optional deklariert.

Neben der Generierung einer Typdefinition aus einem Schema ermöglichen OpenAPI-Tools auch die Erstellung von
Schemata aus bestehenden APIs.
Zum Beispiel kann im Spring Framework für Java das Tool springdoc-openapi ein Schema aus einem 
RestController generieren.
Wenn das generierte Schema vom Frontend zur Erstellung seiner Typdefinition verwendet wird ergibt sich auch
hier eine klare Definition, ohne das das Schema händisch erstellt werden muss.

\subsubsection{Limitationen in Bezug auf Typen}

Obwohl OpenAPI, wie aufgezeigt die Möglichkeit gibt Typen aus definitierten Schnittstellen zu bilden, 
gibt es auch einige wichtige Limitationen bei der Übersetzung.

Ein zentrales Problem ist, dass OpenAPI über das Attribut \texttt{format} zusätzliche semantische Einschränkungen
auf primitive Typen spezifizieren kann, die über die reine Typinformation hinausgehen.
Beispielsweise kann ein Feld vom Typ \texttt{string} das Format \texttt{date-time} tragen, 
was bedeutet, dass nur Zeichenketten im ISO 8601 Zeitformat gültig sind. 
Auch für \texttt{integer} oder \texttt{number} existieren Formate wie 
\texttt{int32}, \texttt{int64} oder \texttt{float}. 
Diese Formate definieren eingeschränkte Wertebereiche oder bestimmte Darstellungsformen.

Das Problem hierbei ist, dass viele Programmiersprachen, insbesondere solche mit struktureller Typisierung wie TypeScript,
diese Formatinformationen nicht direkt in ihre Typensysteme übersetzen können. 
Während das Tooling den Typ korrekt als \texttt{string} oder \texttt{number} übernimmt, 
geht die mit dem Format verbundene Einschränkung dabei verloren. 
Der erzeugte Typ vermittelt also nur einen Teil der in OpenAPI spezifizierten Bedeutung, 
was zu einem Verlust an Präzision führen kann.

Ein weiteres Beispiel betrifft Wertebereiche, 
die über die \texttt{minimum}- und \texttt{maximum}-Eigenschaften eines Schemas definiert werden.
Auch hier kann OpenAPI präzise definieren, dass zum Beispiel ein Zahlenwert zwischen 0 und 100 liegen muss. 
In vielen Zielsprachen lässt sich dies jedoch nicht direkt typisieren, 
da es keine eingebauten Mechanismen für bereichsbegrenzte Typen gibt.
Entsprechend müssen solche Einschränkungen entweder in Validierungslogik separat abgebildet oder manuell dokumentiert werden,
was wiederum potenzielle Inkonsistenzen schafft.

\subsubsection{Herausforderungen automatischer Typgenerierung}

Trotz der offensichtlichen Vorteile automatisierter Typgenerierung, wie der Reduktion manueller Fehler
und der Sicherstellung von Konsistenz zwischen Dokumentation und Implementierung und automatischen Tests\cite{9159071}
bringt dieser Ansatz auch einige Herausforderungen mit sich. 
Ein zentrales Problem stellt der sogenannte Schema-Drift dar:
Wenn sich das OpenAPI-Schema ändert, aber die generierten Typen im Frontend nicht aktualisiert werden, 
kann dies zu schwer auffindbaren Laufzeitfehlern führen. 
Zudem besteht bei generierten Typen die Gefahr einer Fehlinferenz, 
etwa wenn das Schema nicht präzise genug formuliert ist oder vereinfachte Annahmen über optionale Felder getroffen werden.
In größeren Projekten kann es außerdem zu Konflikten zwischen manuell gepflegten und automatisch erzeugten Typdefinitionen kommen,
insbesondere wenn Entwickler*innen lokal Anpassungen vornehmen, 
die beim nächsten Generierungslauf überschrieben werden. 
In solchen Fällen ist eine klare Tooling-Strategie sowie ein bewusstes Versions- und Änderungsmanagement unerlässlich, 
um Konsistenz und Wartbarkeit langfristig sicherzustellen.

\subsection{Persönliche Reflexion}

In meiner bisherigen Arbeit als Softwareentwickler bin ich immer wieder auf die Herausforderungen gestoßen,
die inkonsistente Schnittstellen mit sich bringen. 
Diese Probleme reichen von Missverständnissen in der Kommunikation zwischen verschiedenen Systemen, 
bis hin zu erhöhtem Wartungsaufwand und einer schlechteren Benutzererfahrung. 
Der Umgang mit solchen Schnittstellen kann durchaus mühsam sein, 
da kleine Änderungen oder Fehler schwerwiegende Auswirkungen auf die gesamte Systemarchitektur haben können.

In diesem Kontext habe ich OpenAPI als eine interessante Möglichkeit kennengelernt, 
um Schnittstellen klarer und standardisierter zu gestalten. 
OpenAPI ermöglicht es, Schnittstellen strukturiert und maschinenlesbar zu definieren, was viele Vorteile mit sich bringt.
Durch die Dokumentation von APIs in einer klaren, standardisierten Form können potenzielle Inkonsistenzen frühzeitig erkannt 
und behoben werden, was den Entwicklungsprozess insgesamt effizienter und weniger fehleranfällig macht.

Trotz der vielen Vorteile, die OpenAPI mit sich bringt, bleibt ein gewisser kritischer Blick bestehen. 
Die Definition von APIs erfordert eine sorgfältige Planung und kann bei komplexen Systemen schnell unübersichtlich werden.
Zudem besteht die Gefahr, dass bei einer rein automatisierten Generierung der Schnittstellen die Qualität und Flexibilität leidet,
wenn Entwickler nicht ausreichend in den Prozess eingebunden sind. 
Es bleibt also eine Herausforderung, das Potenzial von OpenAPI voll auszuschöpfen, 
ohne die Kontrolle über die Feinheiten der Schnittstellen und deren Anpassungsfähigkeit zu verlieren.

Insgesamt sehe ich OpenAPI als eine vielversprechende Methode zur Verbesserung der Konsistenz und Wartbarkeit von Schnittstellen,
auch wenn ich die Notwendigkeit für eine kritische und bewusste Implementierung nicht aus den Augen verlieren möchte. 
Die Balance zwischen Automatisierung und manuellem Feingefühl bleibt dabei entscheidend,
um langfristig stabile und leistungsfähige Systeme zu entwickeln.

\section{Zusammenfassung und Ausblick}
%Mit der Zusammenfassung und dem Ausblick wird die Arbeit geschlossen. Dieses Kapitel sollte ca. 10\% vom Umfang der Arbeit ausmachen. 
\subsection{Zusammenfassung}
%Was sind die zentralen Ergebnisse/Erkenntnisse eurer Arbeit?
\subsection{Kritische Reflektion} 
%Betrachtet die Arbeit und euer eigenes Vorgehen kritisch. Was hätte man im Nachhinein anders/besser machen können.
\subsection{Ausblick}
%Wohin geht die Reise? Was könnte man im Anschluss an eure Arbeit tun?

\section{Anhang}
Der Anhang dient als ergänzender Teil der Arbeit und  zählt nicht zur Zeichenbeschränkung.

% optional, falls ihr viele Tabellen nutzt
% \listoftables

% optional, falls ihr viele Abbildungen nutzt
% \listoffigures

\newpage

% Die Liste der verwendeten Literatur, außer Literatur die mit \mybibexclude{qullen-id} markiert wurde
\printbibliography[notcategory=fullcited]

\newpage
\newgeometry{
    right=0.7in,
    left=0.7in            
}

% Literaturliste zum Rechercheprotokoll
% \fullcite{qullen-id} gibt die entsprechende Quelle komplett aus. 
% \mybibexclude{qullen-id} verhindert, dass die entsprechende Quelle im Literaturverzeichnis auftaucht
\begin{tabularx}{\textwidth}{|X|c|}
    \hline
    \textbf{Suchergebnisse} & \textbf{Notizen} \\ \hline \hline
    \fullcite{10.1145/800230.806975}\mybibexclude{10.1145/800230.806975} & Notizen \\ \hline
    \caption{Literaturliste zum Rechercheprotokoll}
\end{tabularx}


\end{document}
