\documentclass{article}
\usepackage{glossaries}
\usepackage{hyperref}
\usepackage{tabularx}
\usepackage{ltablex}
\usepackage[pass]{geometry}

% https://www.overleaf.com/learn/latex/German
\usepackage[utf8]{inputenc}
\usepackage[T1]{fontenc}
\usepackage{csquotes}

\usepackage[backend=biber, style=alphabetic]{biblatex}

\addbibresource{bibfile.bib}
% https://tex.stackexchange.com/questions/111363/exclude-fullcite-citation-from-bibliography#111375
% erstellt einen Befehl, der ermöglicht, dass Quellen mit \fullcite{...} komplett zitiert werden können ohne im Literaturverzeichnis aufgeführt zu werden
\DeclareBibliographyCategory{fullcited}
\newcommand{\mybibexclude}[1]{\addtocategory{fullcited}{#1}}

\usepackage[ngerman]{babel}
% LaTeX trennt bekanntlich Wörter automatisch. Unbekannte Wörter können wie folgt beigebracht werden.
% \usepackage{hyphenat}
% \hyphenation{Mathe-matik wieder-gewinnen}

\usepackage{subcaption}
\usepackage{graphicx}
\graphicspath{{./assets}}

\usepackage{float}
\usepackage{listings}
\usepackage{xcolor}

\definecolor{amber}{rgb}{1.0, 0.49, 0.0}
\definecolor{green}{rgb}{0.4, 1.0, 0.7}

\lstdefinelanguage{Types}{
    keywords={export, interface}
}

\lstdefinelanguage{json}{
    basicstyle=\normalfont\ttfamily,
    numberstyle=\scriptsize,
    breaklines=true,
    showstringspaces=false,
    string=[db]{"},
    stringstyle=\color{green!50!black},
    morestring=[s][\color{black}]{\ \ "}{":},
    keywordstyle=\color{blue},
    keywords={true,false,null,export,interface},
    literate=
     *{0}{{{\color{red}0}}}{1}
      {1}{{{\color{red}1}}}{1}
      {2}{{{\color{red}2}}}{1}
      {3}{{{\color{red}3}}}{1}
      {4}{{{\color{red}4}}}{1}
      {5}{{{\color{red}5}}}{1}
      {6}{{{\color{red}6}}}{1}
      {7}{{{\color{red}7}}}{1}
      {8}{{{\color{red}8}}}{1}
      {9}{{{\color{red}9}}}{1}
      {.}{{{\color{red}.}}}{1}
      {:}{{{\color{gray}{:}}}}{1}
      {,}{{{\color{gray}{,}}}}{1}
      {\{}{{{\color{gray}{\{}}}}{1}
      {\}}{{{\color{gray}{\}}}}}{1}
      {[}{{{\color{gray}{[}}}}{1}
      {]}{{{\color{gray}{]}}}}{1},
}

\lstset{frame=tb,
  language=Types,
  aboveskip=3mm,
  belowskip=3mm,
  showstringspaces=false,
  columns=flexible,
  basicstyle={\small\ttfamily},
  numbers=left,
  numberstyle=\tiny\color{blue},
  keywordstyle=\color{blue},
  commentstyle=\color{dkgreen},
  stringstyle=\color{green},
  identifierstyle=\color{amber},
  breaklines=true,
  breakatwhitespace=true,
  tabsize=4
}

\begin{document}

\begin{titlepage}
    \begin{center}
        \vspace*{1cm}

        \LARGE Seminararbeit im Seminar Trends der Softwaretechnik \\
        \large Sommersemester 2025

        \vspace{1.5cm}

        \huge
        \textbf{Typsichere Schnittstellendefinition von API-Endpunkten}

        \vspace{3cm}

        \large
        \textbf{Nils Derenthal} (7217566) \\
        \def\email{nils.derenthal002@stud.fh-dortmund.de}\tt\href{mailto:\email}{\email} \\
        Informatik Dual

        \vfill
    \end{center}
\end{titlepage}

\tableofcontents

\newpage


\section{Einleitung} 

Die Verlässlichkeit von Web-Anwendungen hat in der heutigen Zeit eine so hohe Relevanz wie noch
nie zuvor und die Wirtschaftlichkeit von Dienstleistenden IT-Unternehmen hängt stark von der Qualität ihrer Software ab.
Ein wichtiger Aspekt in Bezug auf die Verlässlichkeit von Software sind Typsysteme, mit ihrem Versprechen eine Vielzahl
von Fehlern auszuschließen, bevor die Software überhaupt läuft.
Die folgende Arbeit richtet sich an Softwarearchitekten und soll bei der Entscheidungsfindung von Backendtechnologien 
von Webanwendungen helfen, indem das Konzept der Typsysteme sowie deren Eigenschaften erläutert werden.


\subsection{Grundlagen} 

Für die Abeit wird im folgenden angenommen, dass Leser*innen ein Basiswissen in der Webentwicklung besitzen,
auch wenn keine Details diesbezüglich notwendig sind.
Es ist desweiteren hilfreich, wenn Kenntnisse über eine Sprache mit typisierung vorhanden sind
(Beispielsweise Klassen und Objekttypen in Java, Types in TypeScript oder Enums in Rust).
Die Typentheorie muss nicht bekannt sein und alles diesbezügliche wird im Laufe der Arbeit erläutert.

\subsection{Problemstellung} 

Fehler in der Softwareentwicklung sind auch heutzutage noch sehr präsent. Ob in einem Studienprojekt, der Website einer Behörde,
oder dem Betriebssystem eines Millardenkonzerns: Im Alltag trifft man nicht selten auf solche Fehler.

\begin{figure}[H]
  \centering
  \begin{subfigure}[b]{0.4\linewidth}
    \includegraphics[width=\textwidth]{chrome_error}
    \caption{Ein Crash im Chrome-Browser}
  \end{subfigure}
  \hspace{0.5cm}
  \begin{subfigure}[b]{0.4\linewidth}
    \includegraphics[width=\textwidth]{microsoft_error}
    \caption{Ein Fehler während des Microsoft Logins}
  \end{subfigure}
  \caption{Fehler in alltäglichen Anwendungen}
\end{figure}

Kaum eine Software is frei von Fehlern, welche je nach schwere ein großes Betriebsrisiko darstellen, weswegen die Vermeidung von diesen eine hohe Bedeutung
in der Entwicklung hat. 
Ein Ansatz, welcher in den letzen Jahren zunehmend an Bedeutung gewonnen hat sind stärkere Typsysteme, welche vermeintlich dafür sorgen, dass Fehler bereits
in der Entwicklung abgefangen werden können.
Beispiele für jüngere Entwicklungen in diesem Bereich sind Technologien wie TypeScript, welches versucht JavaScript zu typisieren, oder Rust, als eine Sprache die einen Fokus auf
ein ausdrucksstarkes Typsystem legt.

\begin{figure}[H]
  \centering
  \includegraphics[width=0.9\textwidth]{state_of_js}
  \caption{32\% der Befragten in der \textquote{State of JavaScript 2024} Umfrage gaben an das (fehlende) statische Typisierung einer ihrer größten Probleme mit der Sprache sei \cite{Greif_Burel_2024}}
\end{figure}

\subsection{Ziel der Arbeit} 
Was hat der/die Leser*in von dieser Arbeit? Was soll er/sie nach dem Lesen gelernt haben?

Im folgenden sollen Eigenschaften und Vor- wie auch Nachteile von Typsystemen dargestellt werden, um aufzuzeigen 

% ca. 20%

\section{Systematische Literaturrecherche}
%Dieses Kapitel beschreibt in knapper Form das Vorgehen zur Literaturrecherche. Genaueres ist in \cite{keele2007guidelines} zu finden. Dieses Kapitel sollte ca. 20\% vom Umfang der Arbeit ausmachen. Nachfolgend die Leitfragen für die Recherche aus der Einführungsveranstaltung: \\ \\


% Definition eines Rechercheziels - Was möchte nach der Recherche gelernt haben? Schützen Atemschutzmasken vor einer Corona-Infektion? \\ \\
% Definition relevanter Orte/Quellen für Literatur \\
% Bild-Zeitung, Science, The Lancet, … \\ \\
% Definition relevanter Suchbegriffe \\
% Corona, COVID-19, Maske, Mask, FFP-2 \\ \\
% Durchführung der Recherche mit Protokoll \\
% Datum, Quelle, Suchbegriffe, Ergebnisliste  \\ \\
% Bewertung der Ergebnisse \\
% Aktualität, Glaubwürdigkeit, Beitrag zum Rechercheziel

\subsection{Rechercheziel}

Die Zielsetzung der Literaturrecherche besteht darin, den aktuellen Stand der Forschung und Praxis
im Bereich der Typisierung von Web-APIs systematisch zu erfassen und zu analysieren.
Dabei sollen insbesondere folgende Aspekte untersucht werden:

\begin{itemize}
\item Die Rolle und Bedeutung von Typensystemen bei der Modellierung und Beschreibung von API-Schnittstellen, insbesondere im Kontext moderner Web-APIs.
\item Die Anwendung und Umsetzung typentheoretischer Konzepte wie Produkt- und Summentypen in der API-Definition und deren formale Spezifikation durch Standards wie OpenAPI.
\item Die Möglichkeiten und Grenzen automatischer Typgenerierung aus API-Schemata sowie die damit verbundenen Herausforderungen in der Praxis.
\item Bestehende Ansätze und Werkzeuge zur Sicherstellung der Konsistenz zwischen API-Dokumentation und Implementierung.
\end{itemize}

Die Literaturrecherche soll somit eine Grundlage schaffen, 
um die Relevanz und den Nutzen von Typensystemen bei der API-Modellierung zu bewerten,
und die in der Arbeit dargestellten Lösungsansätze einzuordnen.

\subsection{Relevante Quellen für die Literatur}

Zur Quellensuche wurde vor allem Google Scholar herrangezogen, welche die Suche nach wissenschaftlichen Quellen
vereinfacht.
Auf Basis der Suchen wurden die folgenden Quellen / Veröffentlichungsorte von wissenschaftlichen Papieren aufgrund ihrer Relevanz
auf Bezug zu dem Rechercheziel identifiziert.

\subsubsection{ACM Digital Library}

Die ACM Digital Library (ACMDL) ist eine geeignete Quelle für wissenschaftliche Arbeiten 
zu Themen aus der Informatik, da sie peer-reviewte Publikationen aus diesem Bereich bereitstellt.
Sie bietet Zugang zu aktueller Forschung und anerkannten Konferenzen,
die regelmäßig Beiträge zu relevanten Aspekten des Themas veröffentlichen.
Die Inhalte sind wissenschaftlich fundiert und werden von Fachleuten der Community erstellt. \cite{ACM}

\subsubsection{IEEE Xplore Digital Library}

Die IEEE Xplore Digital Library bietet eine verlässliche Grundlage für wissenschaftliche Arbeiten
in diesem Bereich.
Sie umfasst eine große Auswahl an Fachartikeln, Konferenzbeiträgen und technischen Standards 
mit Schwerpunkt auf Informatik und verwandten Disziplinen.
Besonders relevant ist sie aufgrund ihres Fokus auf praxisnahe und anwendungsorientierte Forschung,
die aktuelle Entwicklungen im Themenfeld umfassend abbildet.
Die Inhalte stammen aus etablierten Fachkreisen und unterliegen strengen Qualitätsstandards. \cite{IEEE}

\subsubsection{Blogartikel}

Blogposts bieten eine wichtige Ressource zur schnellen Informationsgewinnung und praktischen Anwendung. 
Entwicklerblogs und technische Beiträge von Unternehmen oder Fachleuten liefern häufig praxisnahe Anleitungen,
Beispielcodes und Erfahrungsberichte, die in der wissenschaftlichen Literatur oft fehlen oder nur stark abstrahiert vorhanden sind. 
Dadurch ermöglichen Blogposts einen unmittelbaren Zugang zu neuen Technologien, Best Practices und aktuellen Entwicklungen.
Gleichzeitig sind jedoch Risiken für die wissenschaftliche Arbeit erkenntlich:
Die Inhalte unterliegen meist keiner formalen Qualitätskontrolle, sind nicht immer langfristig verfügbar und 
reflektieren oft subjektive Sichtweisen einzelner Entwickler*innen.
Für eine fundierte wissenschaftliche Nutzung müssen solche Quellen daher kritisch bewertet und 
durch etablierte Literatur oder offizielle Dokumentationen ergänzt werden.

\subsubsection{Offizielle Dokumentation}

Die offizielle Dokumentation stellt in der Informatik eine der zuverlässigsten Quellen für die Nutzung und Implementierung 
von Technologien dar.
Sie wird in der Regel von den Entwickler*innen der jeweiligen Software oder Plattformen erstellt und bietet detaillierte, 
präzise Informationen zu Funktionen, Parametern und Anwendungsmöglichkeiten. 
Da offizielle Dokumentationen oft direkt aus dem Quellcode und den Designentscheidungen der Entwickler*innen abgeleitet werden, 
gelten sie als die autoritativste und genaueste Informationsquelle.
Zudem bieten sie eine stabile und langfristig verfügbare Referenz,
die regelmäßig aktualisiert wird, um mit neuen Versionen und Änderungen Schritt zu halten.
Ein Risiko lässt sich allerdings darin sehen, dass eine voreingenommene Haltung gegenüber der präsentierten Software existiert.
Deshalb muss auch hier mit Hilfe von unabhängigen Quellen belegt werden wie Beispielsweise Effektiv eine Software ist.

\subsection{Relevante Suchbegriffe}

Auch wenn die Arbeit im Deutschen Formuliert ist, wird ein Großteil der Recherche im Englischen
durchgeführt, da in der Informatik die meiste Literatur Englischsprachig ist, und es so warscheinlicher ist,
dass Quellen gefunden werden.

\subsubsection{Typisierung}

In Bezug auf die Typisierung der Schnittstellen wurden die folgenden
Recherchebegriffe herrausgearbeitet:

\begin{itemize}
\item type systems web APIs
\item static typing API interfaces
\item product and sum types
\item type safety web APIs
\item typing REST bodys
\end{itemize}

Die Suchergebnisse zu diesen Begriffen waren oft Zielführend, allerdings gab es auch viele
Paper, die nicht die gesuchten Informationen geliefert haben.
Einige von den gefundenen Ergebnissen sind zwar älter, können aber trotzdem als
relevant eingestuft werden, da sie sich mit Typensystemen befassen, welche ein
sehr theoretisches und sich nicht änderndes Thema darstellen.
Bei den Quellen zur Webentwicklung wurde darauf geachtet, dass diese aktuelle Entwicklungen darstellen
und nicht veraltet sind.

\subsubsection{OpenAPI}

Auf Basis der Ergebnisse über die Typisierung wurde die Suche nach Informationen zu OpenAPI
mit den folgenden Suchbegriffen durchgeführt:

\begin{itemize}
\item OpenAPI
\item OpenAPI type systems
\item OpenAPI type safety
\item OpenAPI generation
\item OpenAPI endpoints
\end{itemize}

Auch hier hat die Suche sich auf neuere Paper fokussiert, was kein Problem dargestellt hat, da
das spezielle Tooling noch nicht sehr alt ist.

\subsubsection{Herrausforderungen und Risiken von OpenAPI}

Weiterführend wurden für die Suche nach Risiken und Herrausforderungen bei der Nutzung 
von OpenAPI Suchbegriffe definiert:

\begin{itemize}
\item OpenAPI issues
\item OpenAPI schema drift
\item OpenAPI risks
\end{itemize}

Hier wurde besonders darauf geachtet, dass die Paper von einer neutralen Quelle kommen, da
nur so die Integrität sichergestellt werden kann.

\subsection{Bewertung der Quellen}

Die gefundenen Quellen zeigen eine hohe Vertraulichkeit auf, da die meisten von bekannten Universiäten oder
Instituten, sowie in renomierten Magazinen oder Konferenzen veröffentlicht wurden.
Auch auf den Inhalt bezogen wurde eine hohe Qualität der Quellen festgestellt, da
viele Verweise auf vorherige Arbeiten existieren und der Inhalt neutral und Sachgerecht geschrieben ist.

Insgesamt werden die gefundenen Quellen als qualitativ hochwertig befunden und können somit in der Arbeit verwendet werden.


\section{Hauptteil}
Eigentlicher Inhalt der Arbeit mit geeigneter Einteilung in Unterkapitel und ggf. Querbezügen zu anderen Themen des Seminars und der persönlichen Reflexion des Trends (siehe Folien). \\ \\
Der Hauptteil sollte ca. 50\% vom Umfang der Arbeit ausmachen
% ca. 10%

\section{Zusammenfassung und Ausblick}

\subsection{Zusammenfassung}
%Was sind die zentralen Ergebnisse/Erkenntnisse eurer Arbeit?

\subsection{Kritische Reflektion} 

Im Nachhinein lässt sich einiges an Verbesserungspotential sehen. Einer der größten Faktoren ist die Zeitaufteilung,
welche nicht Ideal abgelaufen ist. Einerseits habe ich zu spät angefangen und somit einen Großteil der Arbeit gegen
Ende der geplanten Phase geschrieben, andererseits hätte ich mir konkrete Zwischenziele setzen, und so strukturierter
arbeiten können.

Bei der Recherche hätte ich mir mehr Zeit lassen können und erst Stichpunkte herrausarbeiten können anstatt, dass ich
einfach drauf los schreibe. Dies ist auch darauf Zurückzuführen, dass ich ein persönliches Interesse an dem
bearbeiteten Thema hatte, und so eventuell voreingenommen in die Arbeit gegangen bin.
Hier könnte es Hilfreich sein die Recherche von einer "unwissenderen" Perspektive durchzuführen um neutralere
Ergebnisse zu erhalten.

\subsection{Ausblick}

Basierend auf der Arbeit wäre es möglich verschiedene weitergehende Themen zu behandeln. 
Einerseits kann untersucht werden wie effektiv Typsysteme für den Anwendungsfall der APIs sind, 
anderseits können aber auch Thematisch tiefere Themen beleuchtet werden. 
Beispielsweise könnten weitere Generierungsmöglichkeiten aus OpenAPI Schemas beleuchtet werden 
(Automatisierte API Zugriffe durch KI Agenten, Maschinelles Testen, Dokumentation, etc.).

Auch in Bezug auf Typsysteme und die Typentheorie lassen sich weitere Themenkomplexe finden, wie
beispielsweise Summentypen für Systemnahe Programmierung verwendet werden können oder die Einsatzmöglichkeiten
von Typen als Richtlinie für Generative KI Systeme in der Programmierung.

\section{Anhang}
Der Anhang dient als ergänzender Teil der Arbeit und  zählt nicht zur Zeichenbeschränkung.

% optional, falls ihr viele Tabellen nutzt
% \listoftables

% optional, falls ihr viele Abbildungen nutzt
% \listoffigures

\newpage

% Die Liste der verwendeten Literatur, außer Literatur die mit \mybibexclude{qullen-id} markiert wurde
\printbibliography[notcategory=fullcited]

\newpage
\newgeometry{
    right=0.7in,
    left=0.7in            
}

% Literaturliste zum Rechercheprotokoll
% \fullcite{qullen-id} gibt die entsprechende Quelle komplett aus. 
% \mybibexclude{qullen-id} verhindert, dass die entsprechende Quelle im Literaturverzeichnis auftaucht
\begin{tabularx}{\textwidth}{|X|c|}
    \hline
    \textbf{Suchergebnisse} & \textbf{Notizen} \\ \hline \hline
    \fullcite{10.1145/800230.806975}\mybibexclude{10.1145/800230.806975} & Notizen \\ \hline
    \caption{Literaturliste zum Rechercheprotokoll}
\end{tabularx}


\end{document}
